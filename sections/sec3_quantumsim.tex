\section{Numerical Simulation}
The quantum simulation has been carried out via the split operator method \cite{NoneFeit1982}, relying on the Fourier transform in cartesian coordinates.
Hence, the Frenet Serrat differential equations \crefrange{eq:frenet1}{eq:frenet2} have been solved to retrieve the curves $ \Ga_{n} $ (as in \cref{fig:curves}) and they have the subsequently been used to build the potential in cartesian coordinates to rewrite the Hamiltonian \cref{eq:curvedhamilton} in the Cartesian frame of reference
\begin{equation}
	\label{eq:splitophamilton}
	h = -\partial_{x}^{2}-\partial_{y}^{2} + V_{C}(x,y)
\end{equation}
where now $ V_{C}(x,y) $  is the whole potential in Cartesian coordinates, encompassing both the attractive potential $ V(s,u) $ due to the bending and the harmonic trapping $ V_{\perp}(u) $ of \cref{eq:curvedhamilton}.
We then choose the initial wave function $ \psi_{0} $ to be a travelling 2D Gaussian wave packet with initial momentum $ k_{0} $ in a harmonic waveguide of trapping frequency $ \omega $ as in \cref{eq:initial_state}.
Moreover, the longitudinal and transverse dispersion $ \sigma_{x} $  and $ \sigma_{y} $ have been fixed to 0.05 and $ \frac{1}{2} m\omega $ respectively.
Finally, the mass have been set to $ m = 400\hbar T/R^{2} $.
We will then simulate the time evolution of $ \psi_{0} $ \cref{eq:initial_state}  under the effect of the potential $ V_{C}(x,y) $, to obtain the final state $ \psi_{so} $.
To find the fidelity of this protocol we need to define a reference case, assumed to be ideal. 
In an ideal situation the wave function would evolve as if it was moving along a straight waveguide, hence no mixing between the longitudinal and transverse components would arise.
We refer to this reference wave function as $ \psi_{f} $ and we will evaluate the overlap with $ \psi_{so} $ to finally define the fidelity
\begin{equation}
	\label{eq:fidelity}
	F = |\braket{\psi_{f}|\psi_{so}}|^{2}.
\end{equation}
This approach is based on the assumption that - in first approximation - a quantum particle can be modelled as a classical one and this assumption has been proven to be effective and to yield high level of fidelity ($>99\%$).
But if we really want to compare the effects of the geometry on the fidelity for different curves $ \Ga_{n} $, the variables $ k_{0} $ and $ \omega $ need to be set in order to make the geometry of the system predominant with respect to the other properties.
With this in mind, we have decided to run the simulations for $ \omega T = 3,5,7 $ while the initial momentum has been varied until a significant drop in fidelity is reached and in the next section we will show and explain the results of our calculations.
