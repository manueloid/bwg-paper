\section{Introduction}
The advancements of quantum technologies in  recent years have brought the focus of the scientific community on the need to investigate novel approaches to produce more compact and stable quantum devices.
While the progressive miniaturisation of the devices poses geometrical constraints on the system, an optimal level of control on the quantum particles is required.
Waveguides in particular play a crucial role in the confinement and transmission of quantum states in a wide variety of experiments where the control is usually achieved by increasing the intensity of the trapping.
This approach is clearly energy consuming so it would be advisable to design protocols relying primarily on the geometry of the waveguide and only subsequently focusing on the strength of the trapping.
The effects of the geometry on waveguides are well known and could be quite disrupting as - for example - it has been shown that even the slightest bending in waveguides results in the production of bound states \cite{BoundStatesInGoldst1992}.
Historically, studies relied on the adiabatic approximation, by virtue of which the curvature is supposed to vary gently.
The adiabatic approximation is ineffective in those systems where strong geometrical constraints are in place as, for instance waveguides with sharp bendings \cite{BoundStatesInClark1996, BoundStatesInBittne2013, MultipleBoundCarini1993}.
Recently, the surge of inverse engineered protocols under the name of Shortcuts to Adiabaticity  \cite{ShortcutsToAdGuery2019} (STA) has proven to be extremely effective in systems where the adiabatic approach is deemed to be unapplicabile.
STA offer a framework to control the evolution of a system starting in some initial state to obtain the desired final state by ensuring that selected external parameters meet the boundary conditions.
In the context of waveguides, Gu\'ery-Odelin et al. in the seminal paper \cite{QuantumControlImpens2020} have applied the approach based on STA to design the curvature of sharply bent waveguides to minimise the loss of coherence in travelling quantum particles with excellent results.
The range of applicability of this protocol extends to all systems where matter-wave circuits are employed.
One of the most relevant example of application is atomtronics \cite{RoadmapOnAtomAmico2021}, an emerging field where the information carriers are cold neutral atoms and the applications of which span from quantum interferometry \cite{MagneticallyGuQiLu2017, 80kmomentumSeMcdona2013} to quantum circuital elements \cite{FocusOnAtomtrAmico2017, AdvancesInAtoPepino2021}.
The approach in \cite{QuantumControlImpens2020} is based on the assumption that - in first approximation - a quantum particle can be modelled as a classical one and that by imposing the trajectory it is possible to inverse engineer the curvature satisfying the boundary conditions.
In this paper we will first briefly review the STA-based approach of Gu\'ery-Odelin et al. and we subsequently change the controlling parameters to further investigate the validity of the protocol when the classical approximation loses its validity.
